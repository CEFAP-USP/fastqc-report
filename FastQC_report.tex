\documentclass[a4paper]{article}
\usepackage[utf8]{inputenc}
% \usepackage[T1]{fontenc}
\usepackage{amsmath}
\usepackage{indentfirst}
\usepackage{graphicx}
\usepackage{float}
\usepackage{longtable}


\usepackage{geometry}
\geometry{
a4paper,
total={170mm,257mm},
left=10mm,
top=20mm,
}


\renewcommand{\figurename}{Figura}
\renewcommand{\tablename}{Tabela}


\begin{document}

\begin{figure}[!htb]
    \centering
    \begin{minipage}{.5\textwidth}
        \centering
        \includegraphics[scale = 0.27]{logo_CEFAP.png}
    \end{minipage}%
    \begin{minipage}{0.5\textwidth}
        \centering
        Centro de Facilidades de Apoio à Pesquisa\\
        Universidade de São Paulo - USP
    \end{minipage}
\end{figure}

\begin{center}

\section*{Relatório da corrida}

\end{center}

\section*{Informações técnicas}

SampleSheet elaborada com as informações que constam no formulário de sequenciamento.

% \begin{table}[!ht]
% \centering
% \footnotesize
\begin{tiny}
\begin{longtable}$TABLECOLUMNS$
\caption{Detalhes da corrida realizada no equipamento $EQUIPAMENTO$.}
\label{TabDetalhesCorrida}
\endfirsthead
\hline
% \begin{tabular}{|l|l|l|l|l|l|l|l|} \hline
$TABLECONTENTS$
\end{longtable}
\end{tiny}
% \end{tabular}
% \end{table}

\section*{Relatório da corrida gerado pelo sistema BCL2FASTQ}

\begin{table}[!ht]
\centering
\footnotesize
\caption{Relatório da corrida gerado pelo sistema BCL2FASTQ - $TABLEAHEADER$}
\label{TabRelA}
\begin{tabular}{|l|l|l|} \hline
$TABLEACONTENTS$
\end{tabular}
\end{table}

% \begin{table}[!htbp]
% \centering
% \tiny
\begin{tiny}
\begin{longtable}{|l|l|l|l|l|l|l|l|l|l|l|l|}
\caption{Relatório da corrida gerado pelo sistema BCL2FASTQ - $TABLEBHEADER$}
\label{TabRelB}
\endfirsthead
\hline
% \begin{tabular}{|l|l|l|l|l|l|l|l|l|l|l|l|} \hline
$TABLEBCONTENTS$
\end{longtable}
\end{tiny}
% \end{tabular}
% \end{table}

% \pagebreak

\begin{center}
\section*{Informações selecionadas do FastQC para a LANE $LANE$ $READ$}
\end{center}

\subsection*{Distribuição de qualidade ao longo dos reads}

\begin{figure}[!htbp]
\centering
\includegraphics[scale = 0.42]{$PATH$/per_base_quality.png}
\caption{Distribuição de qualidade ao longo dos reads}
\label{FigQualidadeReads}
\end{figure}

Escala em \% de erros: 10=90\%, 20=9\%, 30=0,9\%, 40=0,09\%.

Coloração de fundo: Vermelho (Ruim) = 0 a 20; Amarelo (Média) = 20 a 28; Verde (Boa) = 28 ou maior.

Gráfico:
Linha vermelha = mediana;
Caixa amarela = intervalo interquartil;
Traços = pontos 10\% e 90\%;
Linha azul = média.

\subsection*{Distribuição de qualidade média dos reads}

\begin{figure}[!htbp]
\centering
\includegraphics[scale = 0.42]{$PATH$/per_sequence_quality.png}
\caption{Distribuição de qualidade média dos reads}
\label{FigQualidadeMediaReads}
\end{figure}

\pagebreak

\subsection*{Conteúdo médio de bases ao longo dos reads}

\begin{figure}[!htbp]
\centering
\includegraphics[scale = 0.42]{$PATH$/per_base_sequence_content.png}
\caption{Composição média de bases ao longo dos reads.}
\label{FigQualidadeMediaReads}
\end{figure}

\section*{Observações}

As métricas contidas neste relatório têm como base os reads gerados na corrida.

Para visualização das métricas da corrida, utilizar o software Illumina Sequence Analyzer Viewer (SAV), disponível no BaseSpace, selecionando a pasta com os resultados da sua corrida.

\bigskip
\textbf{Pedimos aos usuários da Facility GENIAL (\textit{Genome Investigation and Analysis Laboratory}), que por favor, citem o
Núcleo GENIAL do CEFAP-USP em suas publicações.}

\end{document}